\documentclass{beamer}

\usetheme{default}
\usecolortheme{oi}

\usepackage[francais]{babel}
\usepackage[utf8]{inputenc}
\usepackage{multimedia}

%\usepackage[francais]{babel}


\usepackage{listings}
\usepackage{color}
\usepackage{textcomp}
\definecolor{listinggray}{gray}{0.9}
\definecolor{lbcolor}{rgb}{0.9,0.9,0.9}
\lstset{
	backgroundcolor=\color{lbcolor},
	tabsize=4,
	rulecolor=,
	language=java,
        basicstyle=\scriptsize,
        upquote=true,
        aboveskip={1.5\baselineskip},
        columns=fixed,
        showstringspaces=false,
        extendedchars=true,
        breaklines=true,
        prebreak = \raisebox{0ex}[0ex][0ex]{\ensuremath{\hookleftarrow}},
        frame=single,
        showtabs=false,
        showspaces=false,
        showstringspaces=false,
        identifierstyle=\ttfamily,
        keywordstyle=\color[rgb]{0,0,1},
        commentstyle=\color[rgb]{0.133,0.545,0.133},
        stringstyle=\color[rgb]{0.627,0.126,0.941},
}

\author{Élisabeth Henry}
\title{Threads en Java}

\setbeamertemplate{navigation symbols}{}
\begin{document}

\begin{frame}
  \titlepage
\end{frame}


\begin{frame}
  \frametitle{Une petite recette}
  Recette de cookies
  \begin{itemize}
  \item Préparation : 20 minutes
  \item Cuisson : 10 minutes
  \end{itemize}
  \begin{itemize}
  \item Préchauffer le four à 180 degrés
  \item Mélanger le beurre et le sucre en poudre
  \item Ajouter les \oe ufs, 1 pincée de sel et mélanger encore
  \item Dans un autre récipient, mélanger la farine et la levure
  \item Incorporer cette farine au mélange \oe uf-beurre
  \item Ajouter les pépites de chocolat et mélanger
  \item Former des tas de pâte et mettre au four
  \item Laisser cuire 10 minutes
  \end{itemize}
\end{frame}

\begin{frame}
  \frametitle{Programmation de la recette}
Classe Recette
\lstinputlisting{recette.java}
\end{frame}

\begin{frame}
  \frametitle{Programmation de la recette}
Classe Four
\lstinputlisting{four1.java}
\end{frame}

\begin{frame}
  \frametitle{Résultats}
\lstinputlisting{recette1.txt}
  \uncover<2>{
    \begin{alertblock}{}
      \begin{center}
        Pendant que le four préchauffe, on ne fait rien...
      \end{center}
    \end{alertblock}
  }
\end{frame}

\begin{frame}
  \frametitle{Importance du multithreading}
  \begin{itemize}
  \item Il n'est pas toujours pertinent d'exécuter l'ensemble d'un
    programme séquentiellement
  \item $\rightarrow$ ressources bloquantes
    \begin{itemize}
    \item four.prechauffer ();
    \item lecture/écriture sur des périphériques
    \item demande d'entrée de l'utilisateur
    \item attente d'une connexion réseau
    \item ...
    \end{itemize}
  \item $\rightarrow$ utilisation de plusieurs processeurs
  \end{itemize}
\end{frame}

\begin{frame}
  \frametitle{Threads et processus}
  \begin{itemize}
  \item Un processus est un programme en cours d'exécution
  \item Contient :
    \begin{itemize}
    \item des instructions (les lignes de code) ;
    \item de la mémoire (espace d'adressage) ;
    \item des ressources (fichiers ouverts, sockets, ...).
    \end{itemize}
  \item Il est possible d'avoir plusieurs threads dans un processus
    \begin{itemize}
    \item la mémoire, les ressources, le code... sont partagés entre
      les threads d'un même processus ;
    \item ce qui distingue deux threads c'est leur «position» dans le
      code (d'un point de vue technique, la pile et les registres)
    \end{itemize}
  \end{itemize}
\end{frame}


\begin{frame}
  \frametitle{Les threads en Java}
  En Java, il existe deux façons de gérer plusieurs Threads :
  \begin{itemize}
  \item l'interface Runnable ;
  \item la classe Thread, qui implémente cette interface.
  \end{itemize}
  \begin{block}{Interface Runnable}
  L'interface Runnable définit principalement une méthode
  void run () qu'une classe doit implémenter pour lancer un
  nouveau thread. Il est ensuite possible de lancer cette méthode en
  parallèle du reste du programme.
  \end{block}
  \begin{block}{Classe Thread}
    La classe Thread implémente l'interface
    Runnable. Là encore, il est nécessaire d'implémenter la
    méthode void run (). La classe Thread fournit par
    ailleurs un certain nombre de méthodes utiles.
  \end{block}
\end{frame}

\begin{frame}
  \frametitle{Utilisation de la classe Thread}
    \lstinputlisting{exemple_thread.java}
\end{frame}

\begin{frame}{Utilisation de l'interface Runnable}
  \lstinputlisting{exemple_runnable.java}
  
\end{frame}

\begin{frame}
  \frametitle{Retournons à nos cookies}
  Classe four
  \lstinputlisting{four_thread.java}
    \begin{itemize}
    \item run () se charge maintenant d'appeler prechauffer
      (temperature) ;
    \item comme run () ne prend pas de paramètres, et que prechauffer
      demande un entier, il faut le passer avant (ici, dans le constructeur).
    \end{itemize}
\end{frame}

\begin{frame}
  \frametitle{Cookies multithreads}
  Classe Recette
  \lstinputlisting{recette_thread.java}
\begin{block}{Différences}
  \begin{itemize}
  \item Seul la méthode main change (un peu).
  \item Au lieu d'appeler directement four.prechauffer (temperature), il est
  nécessaire de passer la température au constructeur, puis d'appeler
  four.start () (et non pas four.run ()...)
  \end{itemize}

\end{block}
\end{frame}

\begin{frame}
  \frametitle{Résultats}
\lstinputlisting{recette2.txt}
\end{frame}

\begin{frame}
  \frametitle{Synchronisation entre threads}
  Dans cet exemple, on gagne du temps en préchauffant le four
  pendant qu'on commence à préparer
  \begin{alertblock}{MAIS...}
    À aucun moment, on ne vérifie que le four est effectivement chaud
    avant de mettre le plat dedans.
  \end{alertblock}
  Lorsqu'on lance plusieurs threads, on a souvent besoin de les
  synchroniser à un moment donné (en général lorsqu'ils ont fini leur
  traitement).
  
  En effet, il faut vérifier qu'un traitement a bien été accompli
  avant de passer à la suite.

  \begin{itemize}
  \item Thread fournit la méthode join (), qui attend (éventuellement
    un temps limité) qu'un thread ait terminé son exécution.
  \item Lance une InterruptedException si le thread appelant a été interrompu.
  \end{itemize}
\end{frame}

\begin{frame}
  \frametitle{Nouvelle méthode Recette.main}
  \lstinputlisting{recette_thread2.java}
  \begin{itemize}
  \item On s'assure ici que le thread «préchauffement du four» s'est
    terminé avant de continuer.
  \end{itemize}
\end{frame}

\begin{frame}
  \frametitle{Sleep}
  La classe Thread fournit d'autres méthodes :
  \begin{itemize}
  \item static void sleep (int milliseconds) throws InterruptedException
  \item Cette méthode arrête le thread courant pendant le nombre de
    millisecondes passées en paramètre.
  \end{itemize}
  \lstinputlisting{ex_sleep.java}
\end{frame}

\begin{frame}
  \frametitle{Interruptions}
  \begin{block}{Interrupt}
    \begin{itemize}
    \item public void interrupt () throws SecurityException
    \item interrompt un thread existant : thread.interrupt ()
    \item le thread interrompu reçoit une InterruptedException s'il
      est dans une méthode (sleep, join) qui peut la lancer
    \item sinon, il peut savoir qu'il a été interrompu grâce à la
      méthode isInterrupted
    \end{itemize}
  \end{block}
  \begin{block}{isInterrupted}
    \begin{itemize}
    \item public boolean isInterrupted ()
    \item permet de savoir si un thread a été interrompu
    \item si ça renvoie true, le thread doit se terminer proprement
    \end{itemize}
  \end{block}
\end{frame}
\begin{frame}
  \frametitle{Exemple sleep et interrupt}
  \lstinputlisting{interruptexemple.java}
\end{frame}

\begin{frame}
  \frametitle{Autres méthodes de Thread}
  \begin{itemize}
  \item boolean isAlive (): renvoie true si le thread est «vivant»,
    c'est-à-dire s'il a commencé mais n'a pas été terminé ;
  \item static Thread currentThread (): renvoie une référence au
    thread en cours d'exécution ;
  \item final void setDeamon (boolean on): permet de faire de ce
    thread un thread «daemon» (si on est true) ou «utilisateur» (si on
    est false) ;
  \item final boolean isDeamon (): renvoie true si le thread est «daemon» ;
    \begin{block}{Thread daemon}
      Par défaut, les thread créés sont «utilisateur», et le programme ne
      se termine que lorsque tous les threads sont terminés. À
      l'inverse, les threads «daemon» sont terminés dès qu'il n'y a
      plus de threads utilisateur.
    \end{block}
  \end{itemize}
\end{frame}

\begin{frame}
  \frametitle{Ressources partagées}
  \begin{itemize}
  \item Dans les cas simples, des threads différents s'occupent de
    ressources différentes
  \item Lorsque deux threads sont amenés à manipuler les mêmes
    ressources (variables par exemple), il faut faire attention car
    rien n'indique qu'un autre thread n'a pas modifié une variable
    entre deux instructions
  \end{itemize}
\lstinputlisting{exemple_shmem.java}
Dans cet exemple, si un autre thread peut modifier n, rien n'indique
que d'une ligne à l'autre la valeur de n n'ait pas changé $\rightarrow$ les données affichées ne sont pas cohérentes.
\end{frame}

\begin{frame}
  \frametitle{Mot-clé synchronized}
Pour éviter que deux threads n'accèdent à une même variable en même
temps, Java propose le mot-clé synchronized.
    \lstinputlisting{synchronized.java}

Quelque soit le nombre de threads créés, on s'assure ainsi que
  pour un objet monObjet de type MaClasse, les méthodes uneFonction ()
  et uneAutreFonction () ne pourront pas être exécutées en même temps.

Autrement dit, pour accéder à une méthode synchronisée d'un objet, un
thread met un verrou sur cet objet et s'assure ainsi qu'aucune autre
méthode synchronisée ne sera exécutée en même temps. 
\end{frame}

\begin{frame}
  \frametitle{Le retour du mot-clé synchronized}
  Il est aussi possible d'utiliser le mot clé synchronized d'une autre
  façon, en limitant le verrouillage à un seul bloc au lieu de toute
  une méthode.
  \lstinputlisting{synchronized2.java}
\end{frame}

\begin{frame}
  \frametitle{wait () et notifyAll ()}
  Parfois, un thread doit attendre qu'un autre thread ait effectué une
  tâche concernant un objet avant de continuer.

  La classe Object propose des méthodes pour gérer cela :
  \begin{itemize}
  \item void wait () throws InterruptedException : le
    thread va attendre jusqu'à ce qu'un autre thread envoie une
    notification concernant cet objet (cela lance alors
    l'Exception). Avant d'utiliser monObjet.wait (), un thread doit
    avoir le verrou sur cet objet (via synchronized) ; cela a pour
    effet de libérer le verrou.
  \item void notifyAll () : réveille tous les threads qui attendent
    pour utiliser un objet verrouillé. Là encore, avant d'utiliser
    monObjet.notifyAll (), un thread doit avoir le verrou sur cet objet.
  \item void notify () : idem, mais ne réveille qu'un seul thread
    parmi ceux qui attendent. (Déconseillé, car le thread choisi
    dépend de l'environnement)
  \end{itemize}
\end{frame}

\begin{frame}
  \frametitle{Deadlock}
  Imaginons...
  \begin{itemize}
  \item qu'un premier thread cherche à poser un verrou d'abord sur l'objet
    o1, puis sur l'objet o2 ;
  \item qu'un second thread cherche à poser un verrou d'abord sur
    l'objet o2, puis sur l'objet o1.
  \end{itemize}
  Il est alors possible que le premier thread pose un verrou sur
  l'objet o1, pendant que le second thread pose un verrou sur l'objet
  o2. Le premier thread attend alors de pouvoir accéder à l'objet o2,
  tandis que le second thread attend d'accéder à l'objet o1.

  $\rightarrow$ On appelle cela deadlock (interblocage en
  français). Les deux threads se bloquent alors mutuellement. 

  $\rightarrow$ Très difficile à débugger, car deux threads peuvent se
  bloquer mutuellement ou pas en fonction de leur ordre
  d'activation. Il faut donc faire attention à l'ordre dans lequel on
  pose des verrous sur des objets.
\end{frame}

\begin{frame}
  \frametitle{Résumé}
  \begin{itemize}
  \item Pour créer un thread, une classe doit hériter de la classe
    Thread (ou implémenter l'interface Runnable) et redéfinir la
    méthode void run () ;
  \item Pour lancer un thread :\\Thread t = new Thread ();\\t.start ();
  \item t.join () attend la fin du thread t;
  \item Thread.sleep (duree) permet de mettre le thread appelant en veille ;
  \item t.interrupt () envoie une interruption au thread t;
  \item this.isInterrupted () permet de savoir si le thread a été
    interrompu ;
  \item Pour le partage de ressources entre plusieurs threads :
    mot-clé synchronized.
  \end{itemize}
\end{frame}
\end{document}
